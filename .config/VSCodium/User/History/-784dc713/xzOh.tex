\documentclass[10pt,a4paper]{article}
\usepackage[utf8]{inputenc}
\usepackage[german]{babel}
\usepackage[T1]{fontenc}
\usepackage{amsmath}
\usepackage{amsfonts}
\usepackage{amssymb}
\usepackage{csquotes}
\MakeOuterQuote{"}

\author{Marc-Olivier Jufer}
\title{Einführung in der Cybersicherheitspolitik - Übung 1}

\begin{document}
	\maketitle

	\section*{Frage 1}
		Direkt nach dem Stuxnet Angriff haben die unterschiedlichen Akteuren Angst gehabt, dass solche Waffen die "neue Atombombe" waren. In der Realität war Stuxnet nicht so schlimm wie eine Atombombe: es hatte viel weniger Konsequenzen (mindestens kurzfristig) und die Konsequenzen waren mehr über die Zeit verteilt. So ist es für Stuxnet, aber wie schlecht könnten andere Cyberwaffen sein ? Wie die Leute aus der NSA im Film gesagt haben, war Stuxnet nur die Spitze des Eisbergs dessen, was man durch die Kontrolle des Cyberspace erreichen kann. Nach ihren Angaben können Cyberwaffen viel gefährlicher werden und als Beispiel sagen sie, dass wenn man einem Land den Strom abstellt, sterben Leute. Wir können uns vorstellen, dass der Angriff, der sich "nur" gegen die Zentrifugen richtete, auch die Atomkraftwerke angreifen können hätte, und damit eine "neue Chernobyl" verursachen, oder noch schlechten, in der für den Fall, dass es den Angreifern gelingt, eine Kernfusion herbeizuführen (wenn auch nahe an der Science-Fiction). Solche Szenarien passen ziemlich gut zum Kommentar über die Stuxnet Angriff "sie drückten den roten Knopf, aber das Programm unterbrach das Signal", wo im Film erwähnt wurde. \\
		
		Nach den oben genanten Punkten ist das Wort "Cyberwaffe" mehr als angemessen, da solche Werkzeuge Konsequenzen haben können, die mit denen herkömmlicherer Waffen, die in Kriegen eingesetzt werden, vergleichbar sind (Zerstörung von Infrastruktur oder sogar von Menschen). Das Wort "Cyberkrieg" kann so verstanden werden, als es ein Krieg ist, wo nur im Cyberspace geschieht. Im Fakt ist es oft so, das Cyberwaffe komplexe militärische Operationen in der physischen Welt unterstützen. Dafür ist die Ausdruck "Fifth Dimension of Warfare" viel besser geeignet, obwohl wir noch diskutieren könnten, ob Cyberspace wirklich erst die fünfte Dimension ist.\\
		
		
		Anschliessend sei darauf hingewiesen, dass diese Analyse nur dann gut ist, wenn die Informationen, die im Film von verschiedenen privaten und staatlichen Akteuren gegeben werden, korrekt sind, was vermutlich nur teilweise der Fall ist.
	
\end{document}